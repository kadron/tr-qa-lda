\documentclass[landscape,final]{baposter}

\usepackage[utf8]{inputenc}

\usepackage{times}
\usepackage{calc}
\usepackage{graphicx}
\usepackage{amsmath}
\usepackage{amssymb}
\usepackage{relsize}
\usepackage{multirow}
\usepackage{bm}
\usepackage{array}
\usepackage{bibspacing}

\usepackage{graphicx,capt-of}
\usepackage{multicol}

% \usepackage{pgfbaselayers}
\pgfdeclarelayer{background}
\pgfdeclarelayer{foreground}
\pgfsetlayers{background,main,foreground}

\usepackage{helvet}
%\usepackage{bookman}
\usepackage{palatino}

\renewcommand{\refname}{}
\def \figdir {../figures}

\newcommand{\captionfont}{\footnotesize}

\selectcolormodel{cmyk}

\graphicspath{{images/}}

%%%%%%%%%%%%%%%%%%%%%%%%%%%%%%%%%%%%%%%%%%%%%%%%%%%%%%%%%%%%%%%%%%%%%%%%%%%%%%%%
%%%% Some math symbols used in the text
%%%%%%%%%%%%%%%%%%%%%%%%%%%%%%%%%%%%%%%%%%%%%%%%%%%%%%%%%%%%%%%%%%%%%%%%%%%%%%%%
% Format 
\newcommand{\Matrix}[1]{\begin{bmatrix} #1 \end{bmatrix}}
\newcommand{\Vector}[1]{\Matrix{#1}}
\newcommand*{\SET}[1]  {\ensuremath{\mathcal{#1}}}
\newcommand*{\MAT}[1]  {\ensuremath{\mathbf{#1}}}
\newcommand*{\VEC}[1]  {\ensuremath{\bm{#1}}}
\newcommand*{\CONST}[1]{\ensuremath{\mathit{#1}}}
\newcommand*{\norm}[1]{\mathopen\| #1 \mathclose\|}% use instead of $\|x\|$
\newcommand*{\abs}[1]{\mathopen| #1 \mathclose|}% use instead of $\|x\|$
\newcommand*{\absLR}[1]{\left| #1 \right|}% use instead of $\|x\|$

\def\norm#1{\mathopen\| #1 \mathclose\|}% use instead of $\|x\|$
\newcommand{\normLR}[1]{\left\| #1 \right\|}% use instead of $\|x\|$

%%%%%%%%%%%%%%%%%%%%%%%%%%%%%%%%%%%%%%%%%%%%%%%%%%%%%%%%%%%%%%%%%%%%%%%%%%%%%%%%
% Multicol Settings
%%%%%%%%%%%%%%%%%%%%%%%%%%%%%%%%%%%%%%%%%%%%%%%%%%%%%%%%%%%%%%%%%%%%%%%%%%%%%%%%
\setlength{\columnsep}{0.7em}
\setlength{\columnseprule}{0mm}


%%%%%%%%%%%%%%%%%%%%%%%%%%%%%%%%%%%%%%%%%%%%%%%%%%%%%%%%%%%%%%%%%%%%%%%%%%%%%%%%
% Save space in lists. Use this after the opening of the list
%%%%%%%%%%%%%%%%%%%%%%%%%%%%%%%%%%%%%%%%%%%%%%%%%%%%%%%%%%%%%%%%%%%%%%%%%%%%%%%%
\newcommand{\compresslist}{%
\setlength{\itemsep}{1pt}%
\setlength{\parskip}{0pt}%
\setlength{\parsep}{2pt}%
}

%%%%%%%%%%%%%%%%%%%%%%%%%%%%%%%%%%%%%%%%%%%%%%%%%%%%%%%%%%%%%%%%%%%%%%%%%%%%%%
%%% Begin of Document
%%%%%%%%%%%%%%%%%%%%%%%%%%%%%%%%%%%%%%%%%%%%%%%%%%%%%%%%%%%%%%%%%%%%%%%%%%%%%%
% \renewenvironment{list}{\begin{list}}{\end{list}}
\newenvironment{llist}{\begin{list}{$\bullet$}{\leftmargin=1em \itemindent=0em \parskip=0em \parsep=0pt}}{\end{list}}
\begin{document}
\setlength{\bibspacing}{\baselineskip}



%%%%%%%%%%%%%%%%%%%%%%%%%%%%%%%%%%%%%%%%%%%%%%%%%%%%%%%%%%%%%%%%%%%%%%%%%%%%%%
%%% Here starts the poster
%%%---------------------------------------------------------------------------
%%% Format it to your taste with the options
%%%%%%%%%%%%%%%%%%%%%%%%%%%%%%%%%%%%%%%%%%%%%%%%%%%%%%%%%%%%%%%%%%%%%%%%%%%%%%
\typeout{Poster Starts}
\background{
  \begin{tikzpicture}[remember picture,overlay]%
    \draw (current page.north west)+(-2em,-0em) node[anchor=north west] {\hspace{-2em}\includegraphics[height=1.1\textheight]{silhouettes_background}};
  \end{tikzpicture}%
}
\definecolor{skyblue}{rgb}{0.87,0.9,1}
\definecolor{white}{rgb}{1,1,1}
\definecolor{black}{cmyk}{0,0,0.0,1.0}
\definecolor{darkYellow}{cmyk}{0,0,1.0,0.5}
\definecolor{darkSilver}{cmyk}{0,0,0,0.1}
\definecolor{darkblue}{rgb}{0,0,0.7}
\definecolor{lighterryellow}{cmyk}{0,0,0.2,0.0}
\definecolor{lightyellow}{cmyk}{0,0,0.3,0.0}
\definecolor{lighteryellow}{cmyk}{0,0,0.1,0.0}
\definecolor{lightestyellow}{cmyk}{0,0,0.05,0.0}
\begin{poster}{
  % Show grid to help with alignment
  grid=no,
  % Column spacing
  colspacing=1em,
  % Color style
  bgColorOne=white,
  bgColorTwo=white,
  borderColor=blue,
  headerColorOne=skyblue,
  headerColorTwo=black,
  headerFontColor=black,
  boxColorOne=white,
  boxColorTwo=skyblue,
  % Format of textbox
  textborder=rounded,
  % Format of text header
  eyecatcher=yes,
  headerborder=open,
  headerheight=0.12\textheight,
  headershape=rounded,
  headershade=plain,
  headerfont=\Large\textsf, %Sans Serif
  boxshade=plain,
%  background=shade-tb,
  background=plain,
  linewidth=2pt
  }
  % Eye Catcher
  {\includegraphics[height=6em]{figures/Bogazici_University_Logo.png}} % No eye catcher for this poster. If an eye catcher is present, the title is centered between eye-catcher and logo.
  % Title
  {\sf \vspace{5pt}%Sans Serif
  %\bf% Serif
  \vspace{-14pt}
  Kinect Assisted Rat Behaviour Analysis for Experiment Automation
}
  % Authors
  {\sf \Large \vspace{5pt} %Sans Serif
  	\textbf{Çağrı Sofuoğlu, İsmet Burak Kadron and Albert Ali Salah}\\
    Department of Computer Engineering, Bo\u{g}azi\c{c}i University, Istanbul, Turkey\\
    \{cagri.sofuoglu, burak.kadron, salah\}@boun.edu.tr

  }
  % University logo
  {\includegraphics[height=7em]{figures/Bogazici_University_Logo.png}
  }

  \tikzstyle{light shaded}=[top color=baposterBGtwo!30!white,bottom color=baposterBGone!30!white,shading=axis,shading angle=30]

  % Width of left inset image
     \newlength{\leftimgwidth}
     \setlength{\leftimgwidth}{0.78em+8.0em}

%%%%%%%%%%%%%%%%%%%%%%%%%%%%%%%%%%%%%%%%%%%%%%%%%%%%%%%%%%%%%%%%%%%%%%%%%%%%%%
%%% Now define the boxes that make up the poster
%%%---------------------------------------------------------------------------
%%% Each box has a name and can be placed absolutely or relatively.
%%% The only inconvenience is that you can only specify a relative position 
%%% towards an already declared box. So if you have a box attached to the 
%%% bottom, one to the top and a third one which should be in between, you 
%%% have to specify the top and bottom boxes before you specify the middle 
%%% box.
%%%%%%%%%%%%%%%%%%%%%%%%%%%%%%%%%%%%%%%%%%%%%%%%%%%%%%%%%%%%%%%%%%%%%%%%%%%%%%
    %
    % A coloured circle useful as a bullet with an adjustably strong filling
    \newcommand{\colouredcircle}[1]{%
      \tikz{\useasboundingbox (-0.2em,-0.32em) rectangle(0.2em,0.32em); \draw[draw=black,fill=baposterBGone!80!black!#1!white,line width=0.03em] (0,0) circle(0.18em);}}

%%%%%%%%%%%%%%%%%%%%%%%%%%%%%%%%%%%%%%%%%%%%%%%%%%%%%%%%%%%%%%%%%%%%%%%%%%%%%%
  \headerbox{Abstract}{name=abstract,column=0,row=0}{
%%%%%%%%%%%%%%%%%%%%%%%%%%%%%%%%%%%%%%%%%%%%%%%%%%%%%%%%%%%%%%%%%%%%%%%%%%%%%%
\par We present an exploration of a generative model for a QA (Question Answering) system applied to a Turkish corpora. The problem is “answering a natural language question using a relevant document”. Topic modeling using LDA (Latent Dirichlet Allocation) to rank the similarities between question and different passages in the document is a known method to solve this kind of problem as shown in the recent work of Celikyilmaz et. al. Their approach is multilingual however a morphologically rich language like Turkish might benefit from a model that is fine tailored for properties of such languages. \\
\par We propose a model that can handle MRLs (Morphological Rich Languages) better when compared to the original model. This model also takes syntactic properties (stems, inflectional affixes) into account because in MRLs substantial grammatical information, i.e., information concerning the arrangement of words into syntactic units or cues to syntactic relations, is expressed at word level. In the end, we apply both models to a Turkish corpora and compare the results. \\

}
%%%%%%%%%%%%%%%%%%%%%%%%%%%%%%%%%%%%%%%%%%%%%%%%%%%%%%%%%%%%%%%%%%%%%%%%%%%%%%
  \headerbox{Motivation}{name=intro,column=0,below=abstract}{
%%%%%%%%%%%%%%%%%%%%%%%%%%%%%%%%%%%%%%%%%%%%%%%%%%%%%%%%%%%%%%%%%%%%%%%%%%%%%%
	\par One such system currently in use is EthoVision (from Noldus Information Technology). Which provides detailed analysis tools to researcher when tracking lab rats. But one limitation of the EthoVision is its inability to operate under lowly lit or dark environments. \\
	\par Our project aims to fill the gap in functionality by providing researchers with a robust, easy to use tool that can operate at various physical set-ups under all lighting conditions. to this end Kinect was selected as the ideal sensor due to its ability to operate unhindered under dark lighting conditions.\\
	
}

%%%%%%%%%%%%%%%%%%%%%%%%%%%%%%%%%%%%%%%%%%%%%%%%%%%%%%%%%%%%%%%%%%%%%%%%%%%%%%
  \headerbox{Original Model}{name=exp,column=1,span=1}{
%%%%%%%%%%%%%%%%%%%%%%%%%%%%%%%%%%%%%%%%%%%%%%%%%%%%%%%%%%%%%%%%%%%%%%%%%%%%%%
	\par The system we developed is going to be used in an experiment where day/light periods are manipulated to observe changes in the rats biological clocks. Thus we decided the depth images provided by the Kinect made the perfect input for the task, as the data is unaffected by any light emission within human visual range.

\begin{equation}
  \label{eq:model}
    \hat{ \phi}_{wi}^{(z_i)} = \frac{n^{WK}_{w_{i}k}+ \beta}{\sum^W_{j=1}n^{WK}_{w_{j}k}+ W \beta} \quad \hat{\theta}^{(s)} \frac{n^{SK}_{sk} +\alpha}{\sum^K_{j=1}n^{SK}_{sj}+ K \alpha}
\end{equation}

\begin{equation}
  \label{eq:dists}
  \begin{aligned}
    w_i|z_i, \phi_{wi}^{(z_i)} & \sim Discrete(\phi^{(z_i)}), \quad i = 1, ..., W \\
    \phi{(z)}  & \sim  Dirichlet(\beta), \quad z = 1, ..., K \\
    z_i | \theta^{(s_i)} & \sim  Discrete(\theta^{(s_i)}), \quad i = 1, ..., W \\
    \theta^{(s)} & \sim  Dirichlet(\alpha), \quad s = 1, ..., S \\
  \end{aligned}
\end{equation}

	\par The application supports a tracking environment of up to 4 cages, each containing a single subject. The position data gathered is presented to the user in raw form and also in a candle-wick graph describing the speed at which a subject moves within a given period as a indicator of a subject's level of activity. \\
	

  }
%%%%%%%%%%%%%%%%%%%%%%%%%%%%%%%%%%%%%%%%%%%%%%%%%%%%%%%%%%%%%%%%%%%%%%%%%%%%%%
  \headerbox{Proposed Model}{name=results,column=2}{
%%%%%%%%%%%%%%%%%%%%%%%%%%%%%%%%%%%%%%%%%%%%%%%%%%%%%%%%%%%%%%%%%%%%%%%%%%%%%%
\par We are implemented a system for automatically tracking rat locomotor behaviour for experiment automation.\\
\par We explored the advantages of using Kinect over classical RGB cameras since this approach allows for the system to operate without concerns over lighting conditions. This solution was deemed necessary because the product owners, researchers at the psychobiology lab, require the ability to track and observe subjects in both day and night conditions.\\

\includegraphics[scale=1.1]{./figures/model} 
	\vspace{-12pt}
    \captionof{figure}{Topic-Word and Passage-Topic Distributions}
    \label{fig:figure}
\vspace{5pt}
}%
\headerbox{Acknowledgements}{name=ack,column=3,row=0}{
%%%%%%%%%%%%%%%%%%%%%%%%%%%%%%%%%%%%%%%%%%%%%%%%%%%%%%%%%%%%%%%%%%%%%%%%%%%%%%
%%%%%%%%%%%%%%%%%%%%%%%%%%%%%%%%%%%%%%%%%%%%%%%%%%%%%%%%%%%%%%%%%%%%%%%%%%%%%%
\par We are grateful to Caner Derici and Yavuz Nuzumlalı for their help and support. We also want to thank Tunga Güngör and the team working with him on TÜBİTAK Project 113E036 for supplying us the data set and the opportunity to work further on their problem.
}

%%%%%%%%%%%%%%%%%%%%%%%%%%%%%%%%%%%%%%%%%%%%%%%%%%%%%%%%%%%%%%%%%%%%%%%%%%%%%%
  \headerbox{References}{name=references,column=3,below=ack}{
%%%%%%%%%%%%%%%%%%%%%%%%%%%%%%%%%%%%%%%%%%%%%%%%%%%%%%%%%%%%%%%%%%%%%%%%%%%%%%
\begin{tiny}
	\nocite{Spink2001731,Lind2005123,Cangar200853}
    \setlength{\parskip}{-2mm}
    \bibliographystyle{ieeetr}
    \bibliography{ref}
\end{tiny}
}%

\end{poster}%
%
\end{document}
